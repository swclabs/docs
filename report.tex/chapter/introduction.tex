\chapter{Mở đầu}
\section{Giới thiệu đề tài}
\subsection{Bối cảnh}
\par \hspace{\parindent} Ecommerce hay thương mại điện tử là xu hướng của thế giới trong thế kỷ 21. Với sự bùng nổ của Internet, ngày nay số lượng người truy cập Internet đang ngày càng tăng và không ngừng tăng lên. Theo thống kê của We are social Singapore, Việt Nam đứng vị trí thứ 16 trong top 20 quốc gia có số lượng người sử dụng Internet nhều nhất tại châu Á. Tỷ lệ người sử dụng Internet tại Việt Nam là 53\%, châu Á là 46,64\% và trên toàn thế giới là 48,2\%. \\ \par

Với những điều kiện thuận lợi về phát triển internet hiện nay, Người bán hàng đã tìm ra được một phương thức mới để có thể rao bán các sản phẩm của mình và dần trở thành một phương tiện tiếp thị quan trọng. Bên cạnh đó việc sử dụng internet vào quá trình kinh doanh hàng hóa đã trở thành thói quen mua sắm online, mang lại thu nhập khá cao cho người bán. là nơi tiêu biểu trog việc đưa thương mại điện thử vào bán lẻ tiêu dùng, góp phần phát triển kinh tế đất nước.
Kết quả khảo sát của Hội Doanh nghiệp hàng Việt Nam chất lượng cao cho kết quả là xu hướng mua bán online ngày càng rõ rệt, đặc biệt với giới tiêu dùng trẻ. Vào năm 2017, mua sắm online mới chỉ chiếm 0,9\% thì chỉ sau một năm, kết quả khảo sát năm 2018 cho thấy số người tiêu dùng chọn mua online đã gấp 3 lần (2,7\%).\\ \par

Mua sắm trực tuyến cho phép khách hàng trực tiếp mua hàng hoặc dịch vụ từ người bán qua Internet bằng cách sử dụng trình duyệt web. Người tiêu dùng tìm thấy một sản phẩm mình quan tâm bằng cách truy cập trực tiếp trang web của nhà bán lẻ hoặc tìm kiếm trong số các nhà cung cấp khác bằng cách sử dụng công cụ tìm kiếm mua sắm, hiển thị sự sẵn có và giá cả của sản phẩm tương tự tại các nhà bán lẻ điện tử khác. Kể từ năm 2016, khách hàng có thể mua sắm trực tuyến bằng nhiều loại máy tính và thiết bị khách nhau, bao gồm máy tính để bàn, máy tính xách tay, máy tính bảng và điện thoại.\\ \par

Thị trường thương mại điện tử Việt Nam với dân số hơn chín mươi triệu người đã trở thành "miếng mồi ngon" khiến các nhà đầu tư không thể ngồi yên. Theo Bộ Công Thương, năm 2022, quy mô thị trường thương mại điện tử bán lẻ Việt Nam ước đạt 16,4 tỷ USD, chiếm 7,5\% doanh thu hàng hóa và dịch vụ tiêu dùng của cả nước.\\ \par

Thương mại điện tử đã trở thành kênh bán hàng vô cùng quan trọng của nhiều doanh nghiệp, đặc biệt là những doanh nghiệp quy mô vừa và nhỏ. Các nhà bán hàng dần chuyển đổi từ hình thức truyền thống sang "online", vừa tiết kiệm chi phí kho bãi, mặt bằng, lại có thể tiếp cận lượng khách hàng khổng lồ. \par

Năm 2018, thị trường thương mại điện tử (TMĐT) chứng kiến cuộc đua mạnh mẽ của các kênh bán lẻ online như shopee, Tiki, Lazada, Điện máy xanh, Thế giới di động, ... Theo Iprice năm 2022, Shopee đạt 84,5 triệu lượt truy cập web mỗi tháng, Thế giới di động đạt 54 triệu, và Điện máy xanh đạt con số 20 triệu. \\ \par

Xuất phát từ nhu cầu cá nhân nhóm cũng như thực tế nói trên, trong phạm vi môn Đề Cương Luận Văn Tốt Nghiệp, nhóm xin triển khai đề tài "\textbf{Phát triển website thương mại điện tử cho doanh nghiệp kinh
doanh sản phẩm công nghệ}" làm luận văn tốt nghiệp của mình. \par

\subsection{Đối tượng sử dụng hệ thống}

\par \hspace{\parindent} Ở thời điểm hiện tại, các mô hình kinh doanh buôn bán online đang rất phổ biến ở Việt
Nam. Với đời sống kinh tế và công nghệ ngày càng một phát triển, nhu cầu mua sắm cá nhân cũng thay đổi, như từ việc chỉ cần sắm cho mình những thiết bị đủ nghe gọi được đến những thiết bị có thể lướt web và xem phim. Trong thời đại bùng nổ về internet và sự phát triển của logitic, ngành hàng công nghệ trở thành mảnh đất màu mỡ để khai thác. \\\par

Nổi lên trong đó các doanh nghiệp, hộ kinh doanh về công nghệ. Khác với doanh nghiệp thực phẩm truyền thống (có hệ thống sản xuất công nghiệp và hệ thống phân phối sản phẩm cố định), các doanh nghiệp, hộ kinh doanh online, sử dụng web thương mại điện tử sẽ có lợi thế về đa dạng mặt hàng, họ có thể cập nhật, cam kết phân phối những sản phẩm tiên tiến và phổ biến của họ tới khách hàng. Với giao diện thân thiện và điều hướng mượt mà giúp tăng trải nghiệm người dùng, đảm bảo rằng khách truy cập có thể dễ dàng tìm kiếm những gì họ đang tìm kiếm. Trang sản phẩm sẽ cung cấp thông tin rõ ràng và ngắn gọn, bao gồm hình ảnh, mô tả, giá cả và đánh giá từ khách hàng về sản phẩm, tạo sự tin tưởng và đáng tin cậy, giúp khách hàng thực hiện mua sắm trực tuyến một cách an toàn.

\subsubsection*{Đối tượng mà trang web hướng đến}
\textbf{Đối tượng yêu cầu}
\begin{itemize}
    \item \textbf{Khách hàng}
    \begin{itemize}
        \item Đối tượng có thể là công ty, doanh nghiệp, hộ kinh doanh, cá thể kinh doanh ngành hàng thiết bị công nghệ.
        \item Đối tượng có kênh bán hàng online, cung cấp sản phẩm đến khách hàng. Có thể là bán qua website, qua sàn thương mại điện tử: Shopee, Lazada, Tiki,.. hoặc qua mạng xã hội: Facebook, Instagram, Tiktok,..
        \item Quy mô: Đối tượng có quy mô đa dạng. Quy trình quản lý không quá phức tạp, hay mang tính chất công nghiệp.
        \item Đối tượng kinh doanh ngành hàng thiết bị công nghệ Ví dụ: Một doanh nghiệp chuyển phân phối, bán sản phẩm thiết bị về công nghệ, bán hàng qua website và fanpage trên Facebook.
    \end{itemize}
    \item \textbf{Quản trị viên}: Quản trị viên của trang web, người quản lý các mặt hàng đang đăng bán trên website
    % \begin{itemize}
    %     \item Theo dõi kho vận, tình trạng xuất nhập kho ở trang quản trị
    %     \item Xác nhận đơn hàng, phát hành mã giảm giá, theo dõi đơn hàng đã xác nhận.
    % \end{itemize}
\end{itemize}
  Đây là mô hình chung mà hệ thống có kế hoạch hướng tới trong tương lai.

\subsection{Phạm vi đối tượng trong đề tài} \par
Cụ thể, trong phạm vi môn Đề Cương Luận Văn, xa hơn là Luận Văn Tốt Nghiệp, nhóm thực hiện hệ thống quản lý đa kênh cho đối tượng:
\begin{itemize}
    \item Là doanh nghiệp có kênh bán hàng online đa kênh, cụ thể là bán qua website và qua mạng xã hội.
    \item Bán hàng qua website: Xây dựng cho shop một website, tại đó, shop có thể đăng bán, tư vấn sản phẩm cho khách hàng.
    \item Bán hàng qua Facebook: Sử dụng Fanpage để bán hàng.
    \item Shop có quy mô đa dạng, quy trình bán hàng đơn giản.
    \item Shop bán hàng online, có chi nhánh và kho chứa hàng.
    \item Ngành kinh doanh công nghệ: Doanh nghiệp kinh doanh các mặt hàng công nghệ. Ví dụ: PC, laptop, camera, router,...
    \item Hàng hóa kinh doanh là hàng có sẵn, hoặc nhập khẩu từ nước ngoài.
  \end{itemize} \par


\section{Các hệ thống tương tự}
\subsection{Chuỗi hệ thống thế giới di động}
\subsubsection{Giới thiệu}
Thế giới di động được hình thành năm 2004 tại Việt Nam và đã phát triển thành một trong những đối tác lớn của các hãng điện thoại di động và các sản phẩm công nghệ khác. lĩnh vực hoạt động chính của công ty bao gồm: mua bán sửa chữa các thiết bị liên quan đến điện thoại di động, thiết bị kỹ thuật số và các lĩnh vực liên quan đến thương mại điện tử.
Bằng trải nghiệm về thị trường điện thoại di động từ đầu những năm 1990, cùng với việc nghiên cứu kỹ tập quán mua hàng của khách hàng Việt Nam, thegioididong.com đã xây dựng một phương thức kinh doanh chưa từng có ở Việt Nam trước đây. \\
\\
Công ty đã xây dựng được một phong cách tư vấn bán hàng đặc biệt nhờ vào một đội ngũ nhân viên chuyên nghiệp và trang web www.thegioididong.com hỗ trợ như là một cẩm nang về điện thoại di động và một kênh thương mại điện tử hàng đầu tại Việt Nam.Hiện nay, số lượng điện thoại bán ra trung bình tại thegioididong.com khoảng 300.000 máy/tháng chiếm khoảng 15\% thị phần điện thoại chính hãng cả nước.\\
\\
Trung bình một tháng bán ra hơn 10.000 laptop trở thành Nhà bán lẻ bán ra số lượng laptop lớn nhất cả nước.Việc bán hàng qua mạng và giao hàng tận nhà trên phạm vi toàn quốc đã được triển khai từ đầu năm 2007, hiện nay lượng khách hàng mua laptop thông qua website www.thegioididong.com và tổng đài 1900.561.292 đã tăng lên đáng kể, trung bình 5.000 - 6.000 đơn hàng mỗi tháng. Đây là một kênh bán hàng tiềm năng và là một công cụ hữu hiệu giúp các khách hàng ở những khu vực xa mua được một sản phẩm ưng ý khi không có điều kiện xem trực tiếp sản phẩm.www.thegioididong.com là website thương mại điện tử lớn nhất Việt Nam với số lượng truy cập hơn 1.200.000 lượt ngày, cung cấp thông tin chi tiết về giá cả, tính năng kĩ thuật của hơn 500 model điện thoại và 200 model laptop của tất cả các nhãn hiệu chính thức tại Việt Nam.

\subsubsection{Một số điểm mạnh của hệ thống}
\textbf{Yêu Cầu Chức Năng}
\begin{itemize}
    \item Hệ thống ổn định trên các trình duyệt web thông dụng (Chrome, Edge, Brave)
    \item Giao diện người dùng ổn định trên các trình duyệt web điện thoại (iOS, android)
    \item Sản phẩm được phân loại cụ thể
    \item Dễ dàng tìm được các sản phẩm mới
    \item Hiệu suất tải trang web nhanh trên các thiết bị sử dụng mạng di động
\end{itemize}
\textbf{Yêu Cầu Phi Chức Năng}
\begin{itemize}
    \item Hỗ trợ thanh toán qua thẻ tín dụng, thẻ ghi nợ
    \item Tra cứu được đơn hàng thông qua tùy chọn \textbf{Tài khoản \& Đơn hàng} ở thanh Navbar
    \item Hỗ trợ xem giá sản phẩm ở các tỉnh thành trên cả nước.
    \item Hỗ trợ tính năng hỏi đáp với nhân viên chăm sóc khách hàng thông qua tùy chọn \textbf{Hỏi Đáp} trên thanh Navbar
    \item Hỗ trợ tính năng cộng điểm khách hàng khi mua sản phẩm
\end{itemize}

\subsection{Hệ thống bán lẻ FPT shop}
\subsubsection{Giới thiệu}
Công ty cổ phần Bán lẻ Kỹ thuật số FPT – FPT Retail ra mắt thị trường Việt Nam từ năm 2012. Công ty là một thành viên của tập đoàn FPT. Hiện nay, FPT Retail đang sở hữu 2 chuỗi thương hiệu bán lẻ với tổng số 500 cửa hàng trải khắp 63 tỉnh thành trên cả nước (tính đến thời điểm tháng 04/2018). Trong đó:

\begin{itemize}
    \item FPT Shop: Đây là chuỗi bán lẻ các sản phẩm công nghệ gồm: điện thoại di động, máy tính bảng, laptop, phụ kiện công nghệ…. FPT shop là chuỗi hệ thống bán lẻ đầu tiên tại Việt Nam được cấp chứng chỉ ISO 9001:2000 về quản lý chất luợng theo tiêu chuẩn quốc tế. Trên thị trường Việt Nam, công ty đang là chuỗi bán lẻ lớn thứ 2 trên thị trường lĩnh vực bán lẻ hàng công nghệ hiện nay.
    \item F Studio by FPT: Đây là chuỗi cửa hàng được Apple ủy quyền chính thức tại Việt Nam nhằm mục đích kinh doanh các sản phẩm chính hãng của Apple. Công ty cổ phần bán lẻ kỹ thuật số FPT là doanh nghiệp đầu tiên tại Việt Nam có chuổi bán lẻ với đầy đủ các mô hình của một cửa hàng của Apple gồm: Apple Premium Reseller, Apple Authorised Reseller, icorner. Điều này giúp cho khách hàng có được không gian tốt nhất để trải nghiệm những sản phẩm độc đáo, chính hãng tới từ Apple.
\end{itemize}

Bằng những nỗ lực  của riêng mình, FPT Shop đã xây dựng được đội ngũ nhân sự với phong cách làm việc chuyên nghiệp, tận tâm với khách hàng. Bên cạnh đó, công ty cũng đang tiếp tục xây dựng trung tâm kinh doanh trực tuyến hiện đại để khách hàng có thể tìm thấy thương hiệu FPT Shop nhanh chóng trên không gian mạng.

\subsubsection{Một số điểm mạnh của hệ thống}
\textbf{Yêu Cầu Phi Chức Năng}
\begin{itemize}
    \item Hệ thống ổn định trên các trình duyệt web thông dụng (Chrome, Edge, Brave)
    \item Giao diện người dùng ổn định trên các trình duyệt web điện thoại (iOS, android)
    \item Sản phẩm được phân loại cụ thể
    \item Dễ dàng tìm được các sản phẩm mới
    \item Hiệu suất tải trang web nhanh trên các thiết bị sử dụng mạng di động
    \item Hiển thị các nền tảng thanh toán trực tuyến ở trang chủ
    \item Hiển thị tùy chọn \textbf{MUA NGAY} \& \textbf{SO SÁNH} ở mục sản phẩm trong trang chủ
\end{itemize}
\textbf{Yêu Cầu Chức Năng}
\begin{itemize}
    \item Hỗ trợ thanh toán qua thẻ tín dụng, thẻ ghi nợ
    \item Tra cứu được đơn hàng thông qua tùy chọn \textbf{Tài khoản \& Đơn hàng} ở thanh Navbar
    \item Hỗ trợ tính năng hỏi đáp với nhân viên chăm sóc khách hàng trực tiếp thông qua nền tảng nhắn tin thời gian thực zalo
    \item Hỗ trợ tính năng cộng điểm khách hàng khi mua sản phẩm
\end{itemize}

\section{Các tính năng của hệ thống}
\subsection{Stakeholders của hệ thống}
\begin{itemize}
    \item Khách hàng
    \item Nhân viên
    \item Quản trị viên
\end{itemize}
\subsection{Yêu cầu chức năng}
\par Hệ thống là website tích hợp cho phép các doanh nghiệp, cá nhân, hộ kinh doanh công nghệ có thể: 
\begin{itemize}
    \item Bán hàng thông qua website thương mai điện tử.
    \item Đăng bán sản phẩm. 
    \item Quản lý các đơn đặt hàng.
    \item Quản lý số lượng các sản phẩm.
    \item Chat, chăm sóc, xác nhận đơn hàng với khách hàng (qua website hay liên kết với Facebook).
    \item Quản lý nguồn hàng xuất, nhập, tồn kho.
    \item Quản lý và báo cáo tài chính
  \end{itemize}
Đồng thời, website cho phép khách hàng:
  \begin{itemize}
    \item Xem thông tin chi tiết sản phẩm.
    \item Mua hàng trực tuyến.
    \item Thanh toán online qua kênh thứ 3.
    \item Có thể chat với cửa hàng thông qua mạng xã hội Facebook.
    \item Chat với CSKH.
  \end{itemize}
Website cần có giao diện hiện đại, dễ dùng,tương tác cao, chuẩn SEO. Dễ quản lý với cả nhân sự phổ thông.
\subsection{Yêu cầu phi chức năng}
Trải nghiệm người dùng
\begin{itemize}
    \item Hệ thống yêu cầu hỗ trợ đầy đủ trên các trình duyệt phổ biến, (Chrome, FireFox, Safari, Cốc Cốc)
    \item Giao diện hỗ trợ trên nhiều thiết bị di động phổ biến.
    \item Ngôn ngữ giao diện: Tiếng Việt - English.
\end{itemize}
\noindent
Tốc độ hiệu năng \& Kích thước
\begin{itemize}
    \item Thời gian chờ không quá 1s
    \item Tổng dung lượng của các file load về thiết bị của người dùng không quá 200MB
\end{itemize}
\noindent
Tính bền vững
\begin{itemize}
    \item Trung bình số lần truy cập hệ thống thất bại là 2 trong 1000 lần truy cập.
    \item Xác suất hệ thống không khả dụng là dưới 0.05\%
    \item Tỷ lệ xảy ra lỗi là dưới 0.03\%
\end{itemize}
Bảo mật, an ninh
\begin{itemize}
    \item  Tài khoản Admin được nhập sai tối đa 5 lần.
    \item  Cảnh báo nếu như có IP máy chủ khác xâm nhập.
    \item  Trang Web có thể ngăn ngừa tấn công DDOS. 
\end{itemize}
Tính mở rộng hệ thống
\begin{itemize}
    \item Hệ thống phải được thiết kế với mức độ chịu tải cao, chống được DDOS
\end{itemize}
\chapter{Cơ sở lý thuyết \& Công nghệ}
\section{Cơ sở lý thuyết}
\subsection{Thương mại điện tử}
\par \hspace{\parindent} Thương mại điện tử là hoạt động mua bán sản phẩm (hoặc dịch vụ) thông qua các hệ thống điện tử như Internet và các mạng máy tính. Thương mại điện tử còn được gọi là E-commerce, nghĩa là Electronic Commerce. \par

Hoạt động thương mại điện tử tại Việt Nam xuất hiện vào những năm đầu của thập niêm 2010. Các hoạt động thương mại điện tử đã chính thức được pháp luật công nhận vào năm 2013. Cụ thể, vào ngày 15/05/2013, Chính phủ ban hành Nghị định số 52/2013/NĐ-CP về thương mại điện tử, theo đó: "Hoạt động thương mại điện tử là việc tiến hành một phần hoặc toàn bộ quy trình của hoạt động thương mại bằng phương tiện điện tử có kết nối với mạng Internet, mạng viễn thông di động hoặc các mạng mở khác". \par

Thương mại điện tử dựa trên một số công nghệ như chuyển tiền điện tử, quản lý chuỗi dây chuyền cung ứng, tiếp thị Internet, quá trình giao dịch trực tuyến, trao đổi dữ liệu điện tử (EDI), các hệ thống quản lý hàng tồn kho và các hệ thống tự thu nhập dữ liệu. \par

\subsection{Các hình thức giao dịch thương mại điện tử}

Thương mại điện tử hiện đại thường sử dụng mạng World Wide Web là một điểm ít nhất phải có trong chu trình giao dịch, mặc dù nó có thể bao gồm một phạm vi lớn hơn về mặt công nghệ như email, các thiết bị di động như là điện thoại. \par

Có tất cả 9 loại hình thức giao dịch thương mại điện tử: \par
 \begin{itemize}
    \item Chính phủ với Chính phủ (G2G)
    \item Chính phủ với Doanh nghiệp (G2B)
    \item Chính phủ với Công dân (G2C)
    \item Doanh nghiệp với Chính phủ (B2G)
    \item Doanh nghiệp với Doanh nghiệp (B2B)
    \item Daonh nghiệp với khách hàng (B2C)
    \item Khách hàng với Chính phủ (C2G)
    \item Khách hàng với Khách hàng (C2C)
    \item Khách hàng với Doanh nghiệp (C2B)
  \end{itemize}

\par Trong đó có 4 hình thức chúng ta thường nghe nhất là: B2C, B2B, C2C, C2B. \par 

Thương mại điện tử đã làm cho hoạt động thương mại của các doanh nghiệp vượt ra khỏi phạm vi quốc gia và trở thành hoạt động mang tính chất toàn cầu. \par

Hiện nay với sự phải triển của thương mại điện tử đã tạo điều kiện thuận lợi cho người dùng đặt mua sản phẩm, thậm chí sản phẩm đó chỉ được bày bán ở khu vực khác, thậm chí ở nước ngoài. Quá trình đặt mua sản phẩm được thực hiện thông qua vài thao tác đơn giản trên các thiết bị di động công nghệ như laptop, điện thoại, máy tính bảng thông qua tương tác với các website thương mai điện tử. \par
\section{Ngôn ngữ nền tảng}
\subsection{Javascript \& Typescript}
\subsubsection*{Javascript}
JavaScript là một ngôn ngữ lập trình mạnh mẽ, đặc biệt được sử dụng phổ biến trong việc phát triển các trang web và ứng dụng web. Ban đầu, JavaScript được tạo ra để tăng cường tính tương tác trên các trình duyệt web, nhưng ngày nay, nó đã trở thành một trong những ngôn ngữ lập trình quan trọng nhất trên toàn cầu.\\
\\
Với JavaScript, các nhà phát triển có khả năng kiểm soát và thay đổi nội dung của trang web, tạo ra hiệu ứng tương tác, và làm cho trải nghiệm người dùng trở nên phong phú và sinh động hơn. JavaScript không chỉ giúp thay đổi giao diện trang web mà còn cho phép tương tác với các API và dữ liệu từ máy chủ, mở ra rất nhiều cơ hội trong việc xây dựng ứng dụng web đa dạng và mạnh mẽ.\\
\\
Với cộng đồng lớn, JavaScript có nhiều thư viện và framework mạnh mẽ như React, Angular, và Vue.js, giúp việc phát triển ứng dụng trở nên dễ dàng hơn và nhanh chóng hơn. JavaScript đã trở thành một ngôn ngữ không thể thiếu trong ngành công nghiệp công nghệ hiện đại và tiếp tục phát triển mạnh mẽ qua thời gian. \\
\\
Một số tính năng cơ bản của JavaScript bao gồm:
\begin{itemize}
    \item \textbf{Tương tác với DOM (Document Object Model)}: JavaScript cho phép thay đổi cấu trúc, nội dung và kiểu dáng của trang web bằng cách tương tác với DOM. Điều này cho phép thay đổi nội dung trang web mà không cần tải lại trang.

    \item \textbf{Sự linh hoạt}: JavaScript là một ngôn ngữ linh hoạt và dễ học. Nó cho phép các nhà phát triển xây dựng các loại ứng dụng từ những tính năng cơ bản đến các ứng dụng phức tạp.

    \item \textbf{Biểu thức điều kiện và vòng lặp}: JavaScript hỗ trợ các cấu trúc điều kiện như if, else, switch và các vòng lặp như for, while giúp kiểm soát luồng logic của chương trình.

    \item \textbf{Xử lý sự kiện}: JavaScript cho phép xử lý các sự kiện như click chuột, gõ phím, hay di chuột qua các phần tử trên trang, từ đó tạo ra các phản ứng tương ứng.

    \item \textbf{Hàm}: JavaScript cho phép xây dựng và sử dụng hàm, giúp tái sử dụng mã nguồn và tạo ra các module có khả năng độc lập.

    \item \textbf{Đa nền tảng}: Ngôn ngữ này có khả năng chạy trên nhiều trình duyệt và nền tảng khác nhau.
\end{itemize}
Tính linh hoạt và khả năng tương tác mạnh mẽ của JavaScript đã làm cho nó trở thành một trong những ngôn ngữ quan trọng nhất trong ngành công nghiệp công nghệ và đóng vai trò không thể thiếu trong việc xây dựng ứng dụng web hiện đại.
\subsubsection*{TypeScript}
TypeScript là một ngôn ngữ lập trình mã nguồn mở, phổ biến trong việc phát triển ứng dụng web, đặc biệt là khi xây dựng ứng dụng lớn và phức tạp. Nó là một phần mở rộng của JavaScript, cung cấp tính năng kiểu dữ liệu tĩnh và các tính năng lập trình hướng đối tượng mạnh mẽ.\\
\\
Các tính năng chính của TypeScript bao gồm:
\begin{itemize}
    \item Kiểu dữ liệu tĩnh: TypeScript cho phép xác định kiểu dữ liệu cho biến, tham số và hàm, giúp phát hiện lỗi và cung cấp thông tin hữu ích trong quá trình phát triển, giảm thiểu rủi ro lỗi và tăng tính bảo mật của mã nguồn.
    \item Lập trình hướng đối tượng: TypeScript hỗ trợ các khái niệm của lập trình hướng đối tượng như lớp, giao diện, kế thừa, và các tính năng quản lý mã nguồn một cách cấu trúc.
    \item Tính mở rộng của JavaScript: TypeScript là một phần mở rộng của JavaScript, điều này có nghĩa là mã TypeScript có thể được biên dịch thành mã JavaScript tiêu chuẩn, cho phép tích hợp dễ dàng vào dự án hiện có.
    \item Tooling mạnh mẽ: TypeScript đi kèm với các công cụ hỗ trợ như trình biên dịch (compiler) giúp chuyển mã TypeScript thành JavaScript, và các trình soạn thảo thông minh với tính năng gợi ý mã (code suggestion) và phát hiện lỗi (error detection).
\end{itemize}
Hỗ trợ cộng đồng lớn: TypeScript có cộng đồng sử dụng và hỗ trợ lớn, điều này giúp người dùng có thể tìm kiếm thông tin, tài liệu và hỗ trợ từ cộng đồng rộng lớn này.\\
\\
Sự kết hợp giữa khả năng linh hoạt của JavaScript và tính năng mạnh mẽ của kiểu dữ liệu tĩnh trong TypeScript làm cho nó trở thành một công cụ quý giá trong việc phát triển ứng dụng web đa dạng và phức tạp.
\subsection{Go (Golang)}
Go, còn được gọi là Golang, là một ngôn ngữ lập trình mã nguồn mở được phát triển bởi Google. Được ra mắt chính thức vào năm 2009, Go nhanh chóng trở thành một trong những ngôn ngữ lập trình phổ biến và mạnh mẽ.\\
\\
Các điểm đặc biệt của Go bao gồm:
\begin{itemize}
    \item Hiệu suất cao: Go được thiết kế với mục tiêu đảm bảo hiệu suất cao, đặc biệt là trong việc xử lý đồng thời (concurrency) và các ứng dụng có tốc độ cao.
    \item Xử lý đồng thời (Concurrency): Go có một mô hình xử lý đồng thời mạnh mẽ với các Goroutines, giúp dễ dàng xử lý hàng ngàn luồng song song mà không gặp vấn đề về hiệu suất.
    \item Đơn giản và dễ đọc: Cú pháp của Go rất đơn giản và dễ đọc, giúp làm giảm độ phức tạp trong việc phát triển và bảo trì mã nguồn.
    \item Kiến trúc hệ thống: Go được thiết kế để xây dựng các hệ thống lớn và phức tạp một cách dễ dàng. Nó cung cấp các công cụ và thư viện hỗ trợ cho việc xây dựng các ứng dụng mạng, máy chủ và các dịch vụ phân tán.
    \item Bảo mật: Go đặt sự chú trọng vào việc đảm bảo an toàn và bảo mật cho ứng dụng được xây dựng bằng việc áp dụng các tiêu chuẩn cao về bảo mật.
    \item Cộng đồng lớn và hỗ trợ tốt: Go có một cộng đồng người dùng rộng lớn, cung cấp tài liệu phong phú, các thư viện và công cụ hỗ trợ đa dạng.
\end{itemize}
Go đã trở thành một lựa chọn phổ biến cho việc phát triển các ứng dụng backend, hệ thống phân tán, và các ứng dụng có yêu cầu về hiệu suất cao và độ tin cậy. Sự kết hợp giữa hiệu suất, đơn giản và khả năng xử lý đồng thời mạnh mẽ làm cho Go trở thành một ngôn ngữ rất hấp dẫn cho các nhà phát triển.

\section{Khung làm việc}
\subsection{NextJS}
Next.js là một framework phát triển ứng dụng web dựa trên React, mang đến sự kết hợp hoàn hảo giữa linh hoạt, hiệu suất cao và khả năng tối ưu hóa. Với Next.js, việc xây dựng ứng dụng web trở nên dễ dàng hơn bao giờ hết, nhờ vào việc cung cấp các tính năng mạnh mẽ như server-side rendering, static site generation, và hệ thống định tuyến tự động.\\
\\
Khả năng tối ưu hóa hiệu suất của Next.js thông qua code splitting, pre-fetching, và caching giúp tăng trải nghiệm người dùng và SEO. Việc hỗ trợ cả TypeScript và JavaScript cũng làm cho Next.js trở thành lựa chọn linh hoạt cho các dự án phát triển ứng dụng web.\\
\\
Tính năng nổi bật như cơ chế hot-reloading và hỗ trợ API Routes giúp việc phát triển và quản lý ứng dụng trở nên thuận tiện hơn, đồng thời cung cấp cho nhà phát triển các công cụ mạnh mẽ để xây dựng các ứng dụng web chất lượng cao.\\
\\
Các tính năng chính của Next.js bao gồm:
\begin{itemize}
    \item Server-side rendering và Static Site Generation (SSR/SSG): Next.js cho phép tạo ra các trang web sử dụng server-side rendering hoặc static site generation, giúp cải thiện tốc độ tải trang và SEO.

    \item Routing tự động: Next.js cung cấp hệ thống định tuyến tự động mà không cần cấu hình phức tạp, giúp quản lý các đường dẫn và trang dễ dàng.

    \item Tối ưu hóa cho hiệu suất: Bằng cách tận dụng các kỹ thuật như code splitting, pre-fetching và caching, Next.js giúp tối ưu hóa hiệu suất của ứng dụng.

    \item Hỗ trợ TypeScript và JavaScript: Next.js hỗ trợ cả TypeScript và JavaScript, cho phép lựa chọn ngôn ngữ phù hợp với dự án.

    \item Hỗ trợ API Routes: Bạn có thể xây dựng các API endpoint một cách dễ dàng thông qua API Routes của Next.js, tạo ra các endpoint truy cập dữ liệu hoặc thực hiện các chức năng cần thiết cho ứng dụng của bạn.

    \item Cơ chế hot-reloading: Next.js tự động tải lại khi bạn thay đổi mã nguồn, giúp phát triển ứng dụng một cách nhanh chóng và hiệu quả.
\end{itemize}
Next.js là một công cụ mạnh mẽ cho việc phát triển các ứng dụng web với sự linh hoạt, khả năng tối ưu hóa hiệu suất và khả năng tương tác tốt với React, cung cấp cho nhà phát triển các công cụ và tính năng giúp xây dựng các ứng dụng web chất lượng cao và hiệu quả.
\subsection{Gin}
Gin là một framework web mã nguồn mở được viết bằng ngôn ngữ Go (Golang), được thiết kế để cung cấp một cách tiếp cận linh hoạt và hiệu quả cho việc phát triển ứng dụng web. Được xây dựng dựa trên nguyên tắc hiệu suất và đơn giản, Gin tập trung vào việc xử lý HTTP requests và routing một cách nhanh chóng.\\
\\
Các điểm đặc biệt của Gin bao gồm:
\begin{itemize}
    \item Hiệu suất cao: Gin được tối ưu hóa để xử lý lượng lớn các request với tốc độ cao, đảm bảo hiệu suất ổn định trong các ứng dụng có yêu cầu tải cao.

    \item Routing linh hoạt: Gin cung cấp hệ thống định tuyến nhanh chóng và dễ sử dụng, cho phép quản lý các endpoint và xử lý các loại request khác nhau một cách dễ dàng.

    \item Middleware mạnh mẽ: Gin hỗ trợ middleware, cho phép thêm các chức năng xử lý trung gian như xác thực, ghi log, nén dữ liệu, hoặc các thao tác tùy chỉnh khác giữa khi request được nhận và khi response được trả về.

    \item Hỗ trợ xây dựng API: Với cú pháp rõ ràng, Gin là một công cụ mạnh mẽ để xây dựng các API, cho phép tạo ra các endpoint API một cách dễ dàng và linh hoạt.

    \item Cộng đồng hỗ trợ và tài liệu phong phú: Gin có một cộng đồng người dùng lớn, cung cấp tài liệu phong phú, ví dụ và hỗ trợ từ cộng đồng sâu rộng.
\end{itemize}
Gin là lựa chọn ưu việt cho việc phát triển các ứng dụng web với tính linh hoạt, hiệu suất cao và khả năng quản lý routing tốt. Sự kết hợp giữa hiệu suất, cú pháp rõ ràng và khả năng mở rộng làm cho Gin trở thành một công cụ hấp dẫn cho các nhà phát triển ứng dụng web sử dụng ngôn ngữ Go.
\section{Cơ sở dữ liệu và hệ thống}
% \subsection{Gorm}
\subsection{Redis}
Redis, viết tắt của "Remote Dictionary Server," là một hệ thống lưu trữ dữ liệu mã nguồn mở, tập trung vào việc lưu trữ và xử lý dữ liệu trong bộ nhớ. Được sử dụng rộng rãi như một cơ sở dữ liệu, bộ đệm, trình giữ thông điệp, và động cơ xử lý luồng, Redis cung cấp một loạt các tính năng mạnh mẽ.\\
\\
Các tính năng chính của Redis bao gồm:
\begin{itemize}
    \item Cấu trúc dữ liệu đa dạng: Redis cung cấp nhiều cấu trúc dữ liệu như chuỗi, danh sách, bảng băm, tập hợp sắp xếp, bitmaps và nhiều loại khác. Điều này cho phép lưu trữ và xử lý dữ liệu theo nhiều cách khác nhau.

    \item Hiệu suất cao: Với việc lưu trữ dữ liệu trong bộ nhớ, Redis cung cấp hiệu suất cao, cho phép truy cập và xử lý dữ liệu nhanh chóng.

    \item Xử lý giao dịch: Redis hỗ trợ giao dịch, cho phép thực hiện nhiều lệnh thành công hoặc thất bại như một giao dịch duy nhất.

    \item Sao chép không đồng bộ: Redis hỗ trợ sao chép dữ liệu không đồng bộ, giúp trong việc đảm bảo sao lưu dữ liệu một cách hiệu quả mà không làm giảm hiệu suất.

    \item Xử lý thông điệp (Pub/Sub): Redis cho phép xử lý thông điệp giữa các ứng dụng thông qua cơ chế Publisher-Subscriber, tạo ra một cách linh hoạt để giao tiếp và trao đổi thông tin.
\end{itemize}
Với khả năng tương tác từ nhiều ngôn ngữ lập trình và các tính năng mạnh mẽ như hiệu suất cao, cấu trúc dữ liệu đa dạng và khả năng xử lý giao dịch, Redis là một công cụ linh hoạt và mạnh mẽ cho việc xây dựng các hệ thống yêu cầu sự đáng tin cậy và hiệu suất.
\subsection{PostgreSQL}
PostgreSQL: Là một hệ quản trị cơ sở dữ liệu quan hệ (Relational Database
Management System) mã nguồn mở và miễn phí. Đây là một trong những RDBMS
phổ biến nhất trên thế giới và được biết đến với độ tin cậy, hiệu suất và bộ tính năng
của nó.\\
\\
Ban đầu PostgreSQL được phát triển bởi một nhóm sinh viên tại Đại học California, Berkeley vào đầu những năm 1990. Dự án sau đó đã được tiếp quản bởi Nhóm Phát triển Toàn cầu PostgreSQL, một tổ chức phi lợi nhuận chịu trách nhiệm phát triển và bảo trì PostgreSQL.\\
\\
PostgreSQL là một RDBMS mạnh mẽ và linh hoạt có thể được sử dụng cho nhiều
ứng dụng khác nhau. Nó rất phù hợp cho cả cơ sở dữ liệu nhỏ và lớn, đồng thời có thể được sử dụng để lưu trữ nhiều loại dữ liệu, bao gồm văn bản, số, ngày tháng và hình ảnh.\\
\\
Khả năng mở rộng cao và dễ dàng có thể đáp ứng nhu cầu của các ứng dụng đang phát triển. Nó cũng có tính bảo mật cao và bao gồm một số tính năng để bảo vệ dữ liệu khỏi bị truy cập trái phép.\\
\\
Một số tính năng chính của PostgreSQL:
\begin{itemize}
    \item Tuân thủ ACID: PostgreSQL Là một cơ sở dữ liệu tuân thủ ACID, có nghĩa là nó
đảm bảo tính toàn vẹn và nhất quán của dữ liệu.
    \item Kiểm soát đồng thời MVCC (multiversion concurrency control - kiểm soát đồng
thời nhiều phiên bản): Để cho phép nhiều người dùng truy cập cùng một cơ sở dữ
liệu cùng một lúc mà không khóa lẫn nhau.
    \item Tối ưu hóa truy vấn: PostgreSQL bao gồm một số tính năng để tối ưu hóa truy
vấn, chẳng hạn như hỗ trợ lập chỉ mục và lập kế hoạch truy vấn.
    \item Hỗ trợ Transaction: Cho phép người dùng nhóm một chuỗi các thao tác lại với
nhau và thực hiện chúng theo tuần tự, nếu có một thao tác bị thất bại, PostgreSQL
sẽ khôi phục chúng về lại trạng thái trước khi thực hiện chuỗi các hoạt động đó.
    \item Bảo mật: PostgreSQL bao gồm một số tính năng bảo mật, chẳng hạn như xác
thực người dùng, kiểm soát truy cập dựa trên vai trò và mã hóa dữ liệu.
    \item Khả năng mở rộng: PostgreSQL có khả năng mở rộng cao và có thể được mở
rộng với nhiều mô-đun khác nhau, chẳng hạn như tiện ích mở rộng không gian,
tiện ích mở rộng tìm kiếm toàn văn bản và tiện ích mở rộng JSON.
    \item Hỗ trợ vùng địa lý: PostgreSQL cung cấp hỗ trợ cho các tính năng liên quan đến
vùng địa lý và địa lý học. Điều này cho phép bạn lưu trữ và truy xuất các dữ liệu
liên quan đến vị trí, bản đồ và khoảng cách một cách dễ dàng.
    \item Tính năng ghi nhật ký (logging): PostgreSQL có khả năng ghi nhật ký (logging)
để theo dõi các hoạt động của cơ sở dữ liệu, giúp bạn dễ dàng kiểm tra và phát hiện các vấn đề xảy ra trong hệ thống.
    \item Tính năng sao lưu và phục hồi dữ liệu: PostgreSQL có khả năng sao lưu và phục
hồi dữ liệu để bảo vệ dữ liệu của bạn trong trường hợp có sự cố xảy ra.
    \item Tính năng đồng bộ hóa dữ liệu: PostgreSQL có khả năng đồng bộ hóa dữ liệu
giữa các cơ sở dữ liệu khác nhau, giúp bạn quản lý và đồng bộ hóa dữ liệu hiệu quả.

\end{itemize}
\subsection{Docker \& Docker Compose}
Docker là một nền tảng mã nguồn mở giúp đóng gói và chạy ứng dụng trong một môi trường container hóa. Bằng cách sử dụng Docker, người dùng có thể đóng gói ứng dụng cùng với tất cả các phụ thuộc của nó, đảm bảo rằng ứng dụng sẽ chạy một cách đồng nhất trên mọi môi trường, từ máy tính cục bộ đến môi trường sản xuất.\\
\\
Các tính năng cơ bản của Docker bao gồm:
\begin{itemize}
    \item Containerization: Docker sử dụng container để đóng gói ứng dụng và các phụ thuộc của nó. Container tách biệt môi trường chạy của ứng dụng, giúp đảm bảo tính đồng nhất và di động.

    \item Khả năng di động: Containers của Docker có thể chạy trên bất kỳ máy chủ hoặc môi trường nào có Docker cài đặt, đảm bảo tính di động và linh hoạt cho việc triển khai.

    \item Quản lý tài nguyên: Docker cung cấp các công cụ để quản lý tài nguyên của container như CPU, bộ nhớ và lưu trữ.

    \item Tích hợp dễ dàng: Docker có thể kết hợp với các công cụ khác như Kubernetes để tạo ra một hệ thống quản lý container phức tạp.
\end{itemize}
Docker Compose là một công cụ giúp xác định và quản lý nhiều container Docker như một ứng dụng duy nhất. Điều này giúp việc triển khai và quản lý các ứng dụng phức tạp, gồm nhiều dịch vụ hoặc phần tử cần phối hợp với nhau, trở nên dễ dàng hơn.\\
\\
Các tính năng của Docker Compose:
\begin{itemize}
\item Định nghĩa đơn giản: Docker Compose cho phép định nghĩa cấu trúc ứng dụng trong một tệp YAML, bao gồm các dịch vụ, cấu hình, và liên kết giữa chúng.

\item Triển khai đồng nhất: Bằng cách sử dụng Docker Compose, người dùng có thể triển khai cùng một cấu hình ứng dụng trên nhiều môi trường một cách đồng nhất.

\item Quản lý nhiều container: Docker Compose giúp quản lý nhiều container và dịch vụ liên quan của chúng, cung cấp khả năng kiểm soát và tự động hóa quy trình triển khai.
\end{itemize}
Sự kết hợp giữa Docker và Docker Compose giúp tạo ra một môi trường linh hoạt và mạnh mẽ để phát triển, triển khai và quản lý ứng dụng.
\subsection{Kubernetes}
Kubernetes, thường được viết tắt là "K8s," là một hệ thống mã nguồn mở được phát triển bởi Google, giúp quản lý và tự động hóa việc triển khai, mở rộng và quản lý các ứng dụng container. Được thiết kế để làm việc với các container, đặc biệt là Docker containers, Kubernetes cung cấp một cơ chế để tự động triển khai, quản lý và mở rộng các ứng dụng container.\\
\\
Các chức năng chính của Kubernetes bao gồm:
\begin{itemize}
    \item Tự động triển khai và mở rộng: Kubernetes cho phép định nghĩa các ứng dụng, cấu hình và triển khai chúng một cách tự động trên các máy chủ. Hệ thống tự động mở rộng khi có yêu cầu tải cao và thu hẹp khi cần thiết.

    \item Quản lý tài nguyên: Kubernetes quản lý tài nguyên máy chủ, bao gồm CPU, bộ nhớ và lưu trữ, để đảm bảo rằng ứng dụng luôn có tài nguyên cần thiết để hoạt động.

    \item Tự phục hồi và độ tin cậy: Khi có lỗi xảy ra trên một máy chủ, Kubernetes tự động chuyển các container hoặc ứng dụng sang máy chủ khác để đảm bảo tính liên tục của ứng dụng.

    \item Cân bằng tải: Kubernetes cung cấp các cơ chế cân bằng tải để phân phối tải làm việc của ứng dụng đều đặn giữa các container và máy chủ.

    \item Khả năng mở rộng linh hoạt: Kubernetes cho phép mở rộng từ các môi trường on-premises đến cloud và ngược lại, đồng thời hỗ trợ nhiều dịch vụ cloud khác nhau.
\end{itemize}
Kubernetes đóng vai trò quan trọng trong việc tạo ra một môi trường linh hoạt, đáng tin cậy và dễ quản lý cho việc triển khai và quản lý các ứng dụng container, giúp tối ưu hoá hiệu suất và tính linh hoạt trong quản lý ứng dụng.
\subsection{Git \& Gitlab}
Git là một hệ thống quản lý phiên bản phân tán, cho phép các nhà phát triển theo dõi các thay đổi trong mã nguồn khi làm việc trên các dự án phần mềm. Được tạo ra bởi Linus Torvalds, Git giúp quản lý lịch sử của mã nguồn, cho phép người dùng theo dõi sự thay đổi, so sánh các phiên bản và kết hợp các phiên bản khác nhau của mã nguồn một cách linh hoạt và an toàn.\\
\\
GitLab, một platform mã nguồn mở, cung cấp các dịch vụ quản lý mã nguồn dựa trên Git. Nó kết hợp các tính năng của Git với quản lý dự án, quản lý vấn đề, CI/CD và hợp tác nhóm. GitLab cho phép người dùng tạo và quản lý các kho mã nguồn, theo dõi vấn đề, xây dựng và triển khai tự động, cung cấp một nền tảng tích hợp để phát triển phần mềm.\\
\\
Bằng cách kết hợp Git và GitLab, các nhóm phát triển có thể quản lý mã nguồn, theo dõi sự thay đổi, làm việc cùng nhau và triển khai phần mềm một cách hiệu quả. Sự linh hoạt và tính toàn diện của GitLab làm cho nó trở thành một công cụ quan trọng trong quy trình phát triển phần mềm hiện đại.
\subsection{NGINX}
Nginx là một máy chủ web mã nguồn mở rất phổ biến, được thiết kế để xử lý các yêu cầu HTTP và HTTPS. Nó cung cấp nhiều tính năng mạnh mẽ, chủ yếu tập trung vào hiệu suất cao, tốc độ xử lý, và khả năng mở rộng.\\
\\
Các tính năng chính của Nginx bao gồm:
\begin{itemize}
    \item Xử lý đồng thời (Concurrency): Nginx có khả năng xử lý hàng ngàn kết nối đồng thời với tài nguyên ít tốn kém, làm giảm tải và tăng hiệu suất.

    \item Web Server: Nginx hoạt động như một máy chủ web, xử lý yêu cầu HTTP, HTTPS và cung cấp các tính năng như cân bằng tải, bộ đệm và xử lý tĩnh.

    \item Reverse Proxy: Nginx có thể hoạt động như một reverse proxy, điều hướng yêu cầu đến các máy chủ ứng dụng khác một cách hiệu quả.

    \item Cân bằng Tải (Load Balancing): Nginx cung cấp cơ chế cân bằng tải giúp phân phối công việc một cách đều đặn cho nhiều máy chủ, giảm tải và tăng khả năng chịu tải của hệ thống.

    \item Giảm Cấu Hình Phức Tạp: Cú pháp đơn giản và dễ hiểu của Nginx giúp quản lý cấu hình một cách hiệu quả, giảm bớt sự phức tạp trong việc cấu hình máy chủ.

    \item  TLS/SSL Termination: Nginx hỗ trợ quản lý SSL/TLS, cho phép giải mã và mã hóa lại các yêu cầu, cung cấp bảo mật cho ứng dụng.
\end{itemize}
Nginx đã trở thành một công cụ không thể thiếu trong việc xử lý yêu cầu web với hiệu suất cao và tính mở rộng linh hoạt. Sự kết hợp giữa tốc độ, khả năng mở rộng và khả năng cấu hình đơn giản làm cho Nginx trở thành một giải pháp phổ biến cho việc triển khai ứng dụng web và cân bằng tải hệ thống.
\subsection{Amazon Web Service}
Amazon Web Services (AWS) là một nền tảng đám mây hàng đầu toàn cầu được cung cấp bởi Amazon. Được ra mắt vào năm 2006, AWS cung cấp một loạt các dịch vụ đám mây bao gồm tính toán, lưu trữ, cơ sở dữ liệu, IoT, machine learning, công cụ phát triển ứng dụng và nhiều dịch vụ khác, cho phép người dùng triển khai và quản lý các ứng dụng của họ một cách linh hoạt và hiệu quả.\\
\\
Các dịch vụ chính của AWS bao gồm:
\begin{itemize}
    \item EC2 (Elastic Compute Cloud): Dịch vụ máy ảo linh hoạt cho phép thuê và quản lý máy chủ ảo cho việc chạy ứng dụng.

    \item S3 (Simple Storage Service): Dịch vụ lưu trữ đám mây có khả năng mở rộng cho phép lưu trữ và quản lý tệp dữ liệu.

    \item RDS (Relational Database Service): Dịch vụ cơ sở dữ liệu quan hệ quản lý tự động, hỗ trợ nhiều cơ sở dữ liệu như MySQL, PostgreSQL, và Oracle.

    \item Lambda: Dịch vụ tính toán serverless cho phép chạy mã mà không cần quản lý máy chủ.

    \item Elastic Beanstalk: Dịch vụ dễ sử dụng để triển khai và quản lý ứng dụng web một cách tự động.
\end{itemize}
AWS đã trở thành một trong những nhà cung cấp dịch vụ đám mây lớn nhất trên thế giới, được sử dụng rộng rãi bởi doanh nghiệp, công ty khởi nghiệp và cá nhân. Đặc điểm chính là khả năng mở rộng, bảo mật cao, tính linh hoạt và các dịch vụ phong phú đã giúp AWS trở thành một lựa chọn phổ biến cho việc triển khai và quản lý hạ tầng IT.